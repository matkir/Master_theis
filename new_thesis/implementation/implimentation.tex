
\chapter{Planning the project}
\section{Using Generative Adversarial Networks for enhancement and prediction}
Generative Adversarial Networks (GAN) was proposed by Goodfellow in 2014. \cite{GoodfellowGAN} %TODO fix cite
GANs is a specialised network consisting of generally two neural networks working against each other. 
The two networks are often called discriminator and generator. The generators job is to make data similar to the training data, and discriminator has the job if finding fake (generated) data.
%TODO insert picture here?

The GAN presented by Goodfellow 2014\cite{GoodfellowGAN} is a simple adversarial network, and it needs to be built upon to be used. Since we are working with images it is more suitable to use
DC-GAN instead. 

\subsection{DC-GAN}
  DC-GAN is a type of an Adversarial Network that still uses the adversarial approach, like the original GAN, but the two networks are deep convolutional networks. DC-GANs.
\subsection{CC-GAN}
  removing green squares

\section{Training an autoencoder to help the GAN}
An autoencoder (AE) is another form of a machine learning network. The autoencoders task is to  
