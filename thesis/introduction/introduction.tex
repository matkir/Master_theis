\section{Background and Motivation}

Cancer is today the second leading cause of death in the world, only behind cardiovascular diseases~\cite{WHODEATH}. 
It is one of the leading causes of mortality worldwide, with an estimated 9.6 million deaths in 2018~\cite{WHOCANCER}.
%It is defined as a disease that has an abnormal cell growth with the potential to spread into other parts of the body.\cite{WhatIsCancer}
Contrary to normal cells, cancer cells are often invasive, and it will spread if not treated. 
In contrast to many other diseases, cancer does not need to start from a foreign entity such as a bacteria or virus, but it is often from a malfunctioning cell that starts dividing rapidly. 
This cell division can happen when a cell is damaged, by for instance radiation or other factors like specific proteins, or other chemicals. The result is that the cell either has damage in the DNA which contributes to abnormal cell division or the cell division itself malfunctions. In both cases the damage causes the cell to divide uncontrollably. 
Cancer can in some cases form without any external forces. The cell division is not always perfect, and dysfunctional cells might start a rapid division after being created. In most cases, this is not a problem, as most cells self destruct when they cannot operate~\cite{selfdestruction,apoptosis}. 

The risk of getting cancer is also increased by age. As we grow older, our body gets more prone to defective cell division, and for each imperfect division, the chance of getting cancerous cells increases.  
Our own body is designed to detect and remove cells that are prone to divide uncontrollably. Unfortunately, this system is not perfect, and the immune system can in some cases overlook cancerous cells.
In either external or internal cases, cancer is by definition this uncontrollable multiplication.




Because cancer can hit anyone, at any age, without any predispositions, it is a heavily researched area, both in Norway and the rest of the world. Despite being such a researched area, it is still one of the top causes of human death. 
Some types of cancer, like breast cancer, is one of the simpler forms of cancer to treat, and at this point, those kind of cancers are non-fatal in 78\% of the cases in the United Kingdom~\cite{UKCancer}. 
    
Humans can get cancer in every major organ, but some types of cancer are more common than others.    
For instance cancer in the gastrointestinal (GI) tract is such a place, with approximately 40,000 cases each year in the UK~\cite{UKCancerBowel}. There are around 16,000 bowel cancer deaths in the United Kingdom every year, and it is the 2nd most common cause of cancer-related death, accounting for 10\% of all cancer mortalities.

Given the global focus on cancer, research into detection and treatment is highly relevant in modern western society. 
Especially with detection of cancerous areas in the body, the advancement of computer-aided diagnosis (CAD) has helped significantly when it comes to early detection and localisation. In addition to the boom in computing power, machine learning has become prevalent in the past few years, and specifically, deep learning has become a tool in image and video classification both within and outside the medical domain~\cite{NIPS2012_4824,DBLP:journals/corr/SimonyanZ14a,DBLP:journals/corr/SimonyanZ14a,DBLP:journals/corr/HeZRS15,DBLP:journals/corr/SzegedyIV16}. 
With machine learning and CAD, researchers have now the ability to help doctors with the vital task of detecting and classifying anomalies found in medical images and videos.

Earlier projects regarding CAD have shown promising results, giving doctors new tools when looking for cancer in the GI tract.

The first project on CAD that formed the basis for this thesis is the basic EIR system by Riegler et al.~\cite{riegler2016eir}.  The Eir system set the goal of automatically detecting diseases in the GI tract from videoes or images in real time. 

The paper on Mimir by Hicks et al. presents a system to both improve the ``black box'' understanding and assist in the administrative duties of writing an examination report, and in summary helping medical staff with CAD~\cite{25953}. 

The work published by Hicks et al. and Riegler et al. show that deep learning has excellent applications when it comes to computer aided diagnosis, but there is insufficient work into generalising the methods to work on new data. Another problem with deep learning is that the model overfits to the dataset, meaning it learns the internal bias associated with the data provided. 

In this thesis we explore these topics. We will look at methods into how to counteract overfitting of medical data, as well as methods to help generalise models to better adapt to new unseen datasets.

\section{Problem Statement}
\label{cha:problemstatement}
Based on the motivation presented in the previous section, we believe that we still have room for improvement when making CAD. Based on the previous work done in the research area~\cite{25956,25953,riegler2016eir}, we present the following two hypothesises as a basis for the thesis:


\noindent
\begin{hyp} \label{hyp:a}
When classifying images, we will get the best result when we have images with the least amount of sparse information\footnote{Sparse information in the setting of this thesis is images where there are no relevant pixels for the classification, and the area has little to no entropy. A specific example for us is the area around images with RGB values of 0.}. 
Hence, by removing areas with sparse information,
we will see an increase in classification performance compared to not removing the areas.
\end{hyp}

\noindent
Hypothesis \ref{hyp:a} talks about how black or white areas in pictures might create unwanted classification errors, and that by removing those areas might improve the results of the classification. We also mention low entropy areas as part of the hypothesis, though this needs to be tested individually.


\noindent 
\begin{hyp} \label{hyp:b}
When training a classifier, we will get a higher probability of generalisation of our results when removing the dataset-specific artefacts\footnote{Artefacts in the setting of this thesis is parts of images where there are components of the image not containing ´´true pixels'' from the real world. A specific example for us is any overlay put on the medical images, or for instance oversaturated pixels or lens flares.} compared to not removing artefacts.
\end{hyp}

\noindent
Hypothesis \ref{hyp:b} talks about pixels that are not originating from the original image, or pixels that do not represent the real sample. We believe that the removal of these dataset specific artefacts, the machine learning algorithm does not learn to take these areas into account when classifying images, and subsequently learns the real features for the dataset, instead of the artificial features created by the artefacts.
%\todo{fix}


Our objective in this thesis is to explore the following two hypothesises to show their validity.

The hypotheses raise the following questions which we will address:\\

\textit{1. Can the process of redrawing an area with a new more relevant information (we define it as inpainting), of sparse areas in datasets help with training and classification performed by machine learning? If so, how detailed should the inpainting be, and how much should be inpainted?}\\

\textit{2. Can inpainting of dataset-specific artefacts help with the classification of previously unseen data done by machine learning? If so, how detailed should the inpainting be, and how much should be inpainted?}

%\todo{Parantes, disse er kanskje litt for like}
\section{Scope and Limitations}
Based on the hypothesises in section \ref{cha:problemstatement}, the scope of the thesis is to check their validity, both each on their own and their validity together with each other. 
Both our scope and the problem statement is based on medical images taken from the GI tract, and the goal is to see if the hypothesises can, in the end, help with medical image classification.
We want to look at the problem statements on three different datasets, all with different attributes, and three forms of inpainting.
For each of the datasets, we test all three combinations of inpainting with two different inpainting algorithms. 
For the six created datasets plus the base dataset we run two different pretrained transfer learning networks to see the success of the newly created dataset.
In addition to doing this at the size $256 \times 256$ pixels (px), we also do all the tests above at double resolution to check the validity at larger image sizes.

In total, we make fourteen datasets, and we test the first seven a total of 70 times, and the last seven 35 times. Including the base case, we do 105 total tests to check the validity of inpainting.



\section{Research Method}
For this thesis, we have decided to use the Association for Computing Machinery's (ACM) methodology for our research. The article ``Computing as a discipline'' presents the discipline of computing into three main categories~\cite{Denning:1989:CD:63238.63239}. 
\subsection{Theory}
The ``theory'' part of the article is rooted in mathematics and describes the development of a theory. The article describes the four steps of the theory phase as (1) characterise objects of study (definition), (2) hypothesise
possible relationships among them (theorem), (3) determine whether the
relationships are true (proof), and (4) interpret results.\\

In this thesis, we touch upon the theory behind machine learning, more specifically deep learning and convolutional neural networks. We identify the problems regarding overfitting and the lack of generalisability.



\subsection{Abstraction}
The ``abstraction'' part of this thesis is rooted in the experimental scientific method and relates to the investigation of the hypothesis. The four of stages the investigation are: (1) form a hypothesis, (2) construct a model and make a prediction, (3) design an experiment and collect data, (4) analyse results. \\

The experiments done in this thesis falls under this category. Also, we have the hypothesises ( \ref{hyp:a} \& \ref{hyp:b} ) and methodology as part of the abstraction. Based on the hypothesises presented, we created tests to check their validity, of which we were able to either verify or refute the theory presented.



\subsection{Design}
The third part, ``design'', is rooted in engineering
and consists of four steps followed in the construction
of a system to solve the given problem: (1) state requirements, (2) state specifications, (3) design and implement the system, (4) test the system. \\

This category was supported by the finished system able to inpaint images to improve classification accuracy. This system was extensively used throughout the thesis to conduct a plethora of experiments.



\section{Main Contributors}
Throughout this thesis we have developed a system for preprocessing data as a tool to help classification. 
The system can take any mask and recreate the masked area, and can be used both in the removal of dataset specific artefacts as well as a general tool for inpainting. 

In section \ref{cha:problemstatement} we outlined two main questions based on \ref{hyp:a} \& \ref{hyp:b}, that we are able to answer:
\begin{enumerate}


\item \textit{Can the process of redrawing an area with a new more relevant information, we define it as inpainting, of sparse areas in datasets help with training and classification performed by machine learning? If so, how detailed should the inpainting be, and how much should be inpainted?}\\

From our testing, we can see that inpainting sparse areas in datasets help with the classification. Based on our results shown in \cref{bar:cvc12k,bar:cvc356,bar:kvasir}, we can see that most of our models benefit from the inpainting. It is important to note that not all inpainting methods benefitted from inpainting.



\item \textit{Can inpainting of dataset-specific artefacts help with the classification of previously unseen data done by machine learning? If so, how detailed should the inpainting be, and how much should be inpainted?}\\


We have strong evidence supporting our hypothesis concerning inpainting artefacts. As our results in \cref{bar:cvc12k,bar:cvc356,bar:kvasir} shows, all but one of our inpainting methods beat the baseline when inpainting artefacts. By inpainting artefacts, we got at some cases at least a doubling in MCC score.
\todo{MICHAEL OR PAAL:\\ Do i add conferences/publications?}

\end{enumerate}


\section{Outline}
The thesis is organised as follows:

\paragraph{Chapter 2: Background}
We give more background information about medical practice and machine learning.
We talk about how modern hospitals administer colonoscopies and give insight into how we find polyps and remove them. Here, we also present how digital diagnosis is performed in the modern era.  
We give an introduction to machine learning and its uses, both the history and present-day applications. We will look at the most successful type of machine learning, and give a brief tour into how it works, and how it can be applied to medical data.
We round off this chapter by looking at how machine learning and medical colonoscopy can work together to help with the detection of anomalies in the GI tract.

\paragraph{Chapter 3: Methodology }
We describe the methodology by presenting the work we want to do to test the hypothesises we use in the thesis.
We first look into how we can solve our problems by using inpainting and go into detail into the areas we want to remove to test our problem statements. 
After this, we describe a system to review our models, followed by technical details on the programming languages and packages used.
We end the chapter by looking at the two programs we end up with to test our theories. 

\paragraph{Chapter 4: Experiments}
We start by giving a review of the datasets we use to train and evaluate our model, followed by the metrics we use to describe our rate of success.
We go more in detail into the six datasets we make, and the 105 total runs we take to ensure reliable results.
We end this chapter by presenting the inpainting datasets and then presenting the evaluation of the datasets.


\paragraph{Chapter 5: Conclusion}
Finally, we summarise and conclude this thesis.
We also present ideas and suggestions for further studies surrounding
the findings in this thesis and present final remarks about the research.


