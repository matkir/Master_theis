\section{Background and Motivation}
  \subsection{Introduction REM}
    Cancer is, today, the second leading cause of death in the world, only behind cardiovascular diseases.\\  %TODO CITE
    It is one of the leading causes of mortality worldwide, with approximately 14 million new cases in 2012. %TODO  "http://www.who.int/mediacentre/factsheets/fs297/en/ World Health Organization. February 2018. Retrieved 19 April 2018.
    It is defined as a disease that has an abnormal cell growth with the potential to spread into other parts of the body.%TODO https://www.cancer.gov/about-cancer/understanding/what-is-cancer
    Contrary to normal cells, cancer cells are often invasive, and it will spread if not treated. 
    In contrast to many other diseases cancer does not start from a foreign entity (such as a bacteria or virus), but it is often from a malfunctioning cell that starts dividing rapidly. 
    This can happen when a cell is damaged, by for instance by radiation or other factors that damages the DNA, and the resulting damage causes the cell to uncontrollably divide. 
    %TODO mention that cells can become cancerous without external `inpact`
    Especially in the later part of life everyone has the chance of getting cancer, and in fact everyone does. Our own body is designed to detect and remove cells that are prone
    to divide uncontrollably. Unfortunately this system is not perfect, and the immune system can in some cases overlook cells that are cancerous.
	
    \vspace{10px}
	
  \subsection{Statistics on cancer REM}
    The western (or modern) world has been in a battle against cancer, and despite a 
    lot of new cures/innovations it is still one of the deadliest killers in the world. 
    \textit{The most common types of cancer in males are lung cancer, prostate cancer, colorectal cancer and stomach cancer.\cite{stewart2014world}}
    
    \vspace{10px}
    
  \subsection{colorectal cancer REM}
    You can get cancer in every major organ, but some types of cancer are more common than others.	
    For instance cancer in the gastrointestinal tract (GI) is one of the more common places 
    to get cancer. This is just behind x, and it has a mortality rate of x in the first y years. %TODO CITE 
    We often call this 5 year survival rate for z. This is the standard way to measure the life expectancy of a patient diagnosed with cancer. 
	
    \vspace{10px}

  \subsection{polyps REM}
    The colorectal cancer often starts in polyps. 
    Polyps are, polyps do.\\
	
    \vspace{10px}
	
  \subsection{preventative matters and early detection REM}
    \textit{-colonoscopy\\ 
		-mri\\
		-pillcam\\}
    A good way to fight cancer is to detect and remove it early, or some times remove areas with a high chance of getting cancer.
    We classify cancer in to x stages, and the stage the patient are in often determines the chance you have for survival. 
    In general, the earlier you find the cancer, the more likely it is that the patient will survive. 
    And as mentioned above, the colorectal cancer often starts in these polyps. A crucial stage to prevent cancer lies in the 
    early removal of there polyps.
    Reports shows x about this %TODO find Reports
	
    *4 stages maybe?
    *early detection
    *survival rate
	
		
    Because of this the ability to find, and remove colorectal polyps is great for preventing cancer in the GI tract. 
	
	
    \vspace{10px}
	\textbf{colonoscopy/Ontonoscopy}
	In the most common way to look for polyps in the GI tract is to use a medical team, and perform a colonoscopy or Ontonoscopy
	colonoscopy is preformed with a camera-stick that is inserted in to the GI tract through the patients anus.\\
	Onoskopy is the same procedure, only the camera is inserted orally. \\
	
	\textbf{Advantages}
	  \begin{itemize}
	    \item Accuracy: The use of a camera controlled by the doctor gives him/her the opportunity to stop at any anomalies.
	    \item Quick results: Since the doctor is doing the procedure the result is given live.
	  \end{itemize}

	\vspace{5px}
	\textbf{Disadvantages}
	  \begin{itemize}
	    \item Expensive: The cost of the doctor and the nurses needed is often high, especially on a routine check.
	    \item Invasion of privacy: Getting an Colonoscopy or Onoskopy is a %TODO
	  \end{itemize}
	
    \vspace{10px}
	\textbf{MRI}
	MRI (Maggnetic stuff) is the act of taking pictures blabla blabla\\
	MRI (Maggnetic stuff) is the act of taking pictures blabla blabla\\
	MRI (Maggnetic stuff) is the act of taking pictures blabla blabla\\
	%TODO MRI VS CT
	\textbf{Advantages}
	  \begin{itemize}
	    \item This is why mri is good %TODO.
	    \item This is why mri is good %TODO.
	  \end{itemize}

	\vspace{5px}
	\textbf{Disadvantages}
	  \begin{itemize}
	    \item This is why mri is bad %TODO.
	    \item This is why mri is bad %TODO.
	  \end{itemize}
    \vspace{10px}
	\textbf{pillcam}
	In the last 3-4 years there have been testing and development on the pillcam project EIR. Machine learning has, through 
	many of the earlier projects, got the detection rate for the polyps up to x\% %TODO cite
	
	\textbf{Advantages}
	  \begin{itemize}
	    \item This is why pillcam is good %TODO.
	    \item This is why pillcam is good %TODO.
	  \end{itemize}

	\vspace{5px}
	\textbf{Disadvantages}
	  \begin{itemize}
	    \item This is why cam is bad %TODO.
	    \item This is why pillcam is bad %TODO.
	  \end{itemize}
 
	
	
    \vspace{10px}

  \subsection{Simulas contribution to the pillcam project REM}
    Simulas EIR
		

		
    * CAD ACD (computer aided diagnosis, Automated computer diagnosis)

    \vspace{10px}
	
\section{Goal / Problem}
  \subsection{pillcam project has lots of data, can be used to train an unsupervised network REM}
    The video sequence from the pillcam can last several hours resulting in thousands of images, combined with colonoscopy images
    we have over 60 000 unlabelled images at our disposal. 
    %TODO Something about the amount of data that exists
    %TODO The time to label data is often a big cost in time and manpower
    
  \subsection{Use Unsupervised learning as a pre-processing tool REM}
    The act of finding an algorithm that can enhance the training data. 
    Either through removing artifacts or virtually enhancing resolution.
    
  \subsection{use Unsupervised-NN/GAN for image enhancements so that a NN can train better REM}
    
    
    
    
    * Now that we got a lot of tests, why not unsupervised
    As mentioned, simula research centre has done a lot of testing on the pillcam project.
		  
		* We know that we can get some results using a neural network
		* Can this be done unsupervised?
		* Can it be done in a fashion that is better than S-ML
		 
		 REM

		
\section{Scope and Limitations}
  \subsection{Use Unsupervised NN to find polyps REM}
  \subsection{Use Unsupervised NN for pre-processing REM}
		* Something about earlier research already got far, so the scope is mainly unsupervised deep learning.
		* (and how to generalise it?)
		*REMegression	
		

\section{Research method}
\section{Related work}
\section{Outline}
  The rest of the thesis is structed as follows:
	
  \textbf{Chapter 2 - Background}\\
	*talk about cancer
	*talk about machine learning.
	*how to use ML on the pillcam video?
  \textbf{Chapter 3 - Me doing stuff}\\
  \textbf{Chapter 4 - Me got and present result}\\
  \textbf{Chapter 5 - Me saying result was good A+}\\
