\section{Background and Motivation}

\textbf{Topic, challenge, solution}\\
Cancer is today, the second leading cause of death in the world, only behind cardiovascular diseases.  %TODO CITE
It is one of the leading causes of mortality worldwide, with approximately 14 million new cases in 2012~\cite{WHOCANCER}.
%It is defined as a disease that has an abnormal cell growth with the potential to spread into other parts of the body.\cite{WhatIsCancer}
Contrary to normal cells, cancer cells are often invasive, and it will spread if not treated. 
In contrast to many other diseases, cancer does not start from a foreign entity (such as a bacteria or virus), but it is often from a malfunctioning cell that starts dividing rapidly. 
This cell division can happen when a cell is damaged, by for instance by radiation or other factors specific proteins, or other chemicals. The result is that the cell either has damage in the DNA which contributes to abnormal cell division or the cell division itself malfunctions. In both cases the damage causes the cell to divide uncontrollably. 
Cancer can in some cases form without any external forces. The cell division is not always perfect, and dysfunctional cells might start a rapid division after being born. In most cases, this is not a problem, as most cells self destruct when they cannot operate\todo{cite}. 

Another factor that increases the risk of cancer is age. As we grow older, our body gets more prone to defective cell division and for each imperfect division the chance of getting cancerous cells increases.  
Our own body is designed to detect and remove cells that are prone to divide uncontrollably. Unfortunately, this system is not perfect, and the immune system can in some cases overlook cells that are cancerous.
In either external or internal cases, cancer is by definition this uncontrollable multiplication.




\paragraph{Statistics on cancer}
Since cancer is, in practice, just a cell division error, it has the opportunity to hit anyone, at any age. Because of this, it is a heavily researched area, both in Norway, and the rest of the world.
In 200X we spent 22Y million dollars worldwide on cancer research. 
Despite being such a researched area, it is still one of the top causes of human death. 
Some types of cancer, like \todo{cancer} cancer, is one of the simpler forms of cancer to treat, and at this point, those kind of cancers are non-fatal. 
\textit{The most common types of cancer in males are lung cancer, prostate cancer, colorectal cancer and stomach cancer~\cite{stewart2014world}.}
    
\paragraph{Colorectal cancer}
Humans can get cancer in every major organ, but some types of cancer are more common than others.    
For instance cancer in the gastrointestinal tract (GI) is one of the more common places to get cancer. Colorectal cancer is just behind x as the most common cancer for men, and it has a mortality rate of x in the first y years. %TODO CITE 
We often call this 5-year survival rate for z. Z is the standard way to measure the life expectancy of a patient diagnosed with cancer. 

\paragraph{CAD}

The work presented in this thesis builds upon the work done by the following publications.


EIR - A Medical Multimedia System for Efficient
Computer Aided Diagnosis by Riegler \todo{cite PhD Michael.}

The thesis of Hicks and subsequent publications made by Hicks et al. is the foundation for this thesis. In Hicks's thesis, the author gives an introduction into the 'black box' that is machine learning and tries to decode how the medical images affect the neural network's decisions. 
The thesis shows how the use of saliency maps can help with the understanding of deep convolutional networks, and it showed how dataset specific artefacts could affect the classification.


The paper, x by Hicks et al. show how removing the dataset-specific artefacts referenced in his thesis gave a better accuracy when it came to classifying images from the GI tract.


The papers submitted by Pathak et al. and Denton et al. and their work on using Generative Adversarial Networks with regards to inpainting are implemented and improved for this dataset-specific task.

Denton et al. in the paper \textit{SEMI-SUPERVISED LEARNING WITH
CONTEXT-CONDITIONAL GENERATIVE
ADVERSARIAL NETWORKS} shows in detail how to use neural networks with an adversarial approach to inpaint areas in images. In their paper, they propose an Encoder-Decoder approach with realistic looking inpaining results.


Pathak et al. shows, in the publication \textit{Context Encoders: Feature Learning by Inpainting} a network and results close the work done by Denton et al. Using a similar Encoder-Decoder network as Denton et al. they achieved similar results inpainting areas in images from the Imagenet dataset\todo{refimagenet}

Our work with Generative Adversarial Networks in this thesis has drawn inspiration from both papers and improved the work to better suit the medical image domain.



\paragraph{Generalisable model is needed}
Though high score not generalisable

\paragraph{Our solution to the problem}
We propose systems to fix this. 
    
    
\section{Problem statement}
Based on the introduction presented in the previous section, we indicate that we still have room for improvement when making computer-aided diagnosis. 
Based on the previous work done in the research area, we \todo{more}


\noindent
\begin{hyp} \label{hyp:a}
When classifying images, we will get the best result when we have images with the least amount of sparse information. 
Hence by removing areas with sparse information,
we will see an increase in classification performance compared to not removing areas.
\end{hyp}

\noindent
and

\noindent 
\begin{hyp} \label{hyp:b}
When training a classifier, we will get a higher
mode of generalisation of our results when removing the dataset
specific artefacts compared to not removing artefacts.
\end{hyp}
\vspace{5px}


\section{Research Method}
For this thesis, we have decided to use the Association for Computing Machinery's (ACM) methodology for our research. The article ``Computing as a discipline'' presents the discipline of computing into three main categories~\cite{Denning:1989:CD:63238.63239}. 
\begin{itemize}
\item Theory
\item Abstraction
\item Design
\end{itemize}
Our work falls partly under all three categories.

\paragraph{Theory}
In this thesis, we touch upon the theory behind machine learning. More specifically deep learning and convolutional neural networks. We identify the problems regarding overfitting and the lack of generalisability.

\paragraph{Abstraction}
The experiments done in this thesis falls under this category. In addition, we have the hypothesises and methodology as part of the abstraction.

\paragraph{Design}
The thesis, except for the published papers, does not fall as much under design. Though future work would be to implement the system into SAGA\todo{citesaga}, the majority of the work only show the theoretical results. 





\section{Main contributors and related work}



Here we will talk about what we achieved.














\section{Outline}
The thesis is organised as follows:

\paragraph{Chapter 2: Background}
We give more background information about medical practice and machine learning.
We talk about how modern hospitals administer colonoscopies and give insight into how we find polyps and remove them. Here we also present how these hospitals perform (DIGITAL DIAGNOSIS), and the potions that are on the market at the present date. 
We give an introduction to machine learning and its uses, both the history and present-day applications. We will look at the most successful type of machine learning, and give a brief tour into how it works, and how it can be applied to data.
We round off this chapter by looking at how machine learning and medical colonoscopy can work together to help with the detection of anomalies in the GI tract.

\paragraph{Chapter 3: Methodology }
We describe the methodology by presenting the datasets we use in this thesis, the methods used when inpainting images, and the design of the experiments and programs. 
We will first look at the Kvasir dataset and go into detail about the creation and attributes with the dataset. We will look at the CVC 356 and the CVC 12k dataset and compare them to the Kvasir dataset.
After discerning the dataset-specific artefacts from the datasets, we look at the options for inpainting the dataset based on the artefacts we want to remove. 
Lastly, we look into the whole process from preprocessing to classifying.


\paragraph{Chapter 4: Experiments}
We give a review of the metrics we use to evaluate our results, followed by the results of the inpainting and classification.


\paragraph{Chapter 5: Conclusion}
 Finally, we summarise and conclude this thesis.
We also present ideas and suggestions for further studies surrounding
the findings in this thesis and present final remarks about the research.

