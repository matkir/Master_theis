\documentclass[a4paper,english]{ifimaster}

\usepackage[utf8]{inputenc}
\usepackage{babel,duomasterforside}
\usepackage{csquotes}
\usepackage{hyperref}

\usepackage[backend=biber,style=authoryear]{biblatex}
\addbibresource{bibliography.bib}





\title{Learning}
\subtitle{something something dark side}
\author{Mathias Kirkerød}

\begin{document}
\duoforside[dept={Department of Informatics},
program={Informatics: Language and Communication},
long]

\frontmatter{}
\chapter*{Abstract}

\tableofcontents{}
\listoffigures{}
\listoftables{}

\chapter*{Preface}

\mainmatter{}
\part{Introduction}
	\section{Motivation}
		Something about machine learning 
	\section{Goals?}
		What we want to achieve

		
		
		
		
		
		
		
		
		
		
		
		
		
		
		
		
		
		
		
		
		
		
		
		
		
		
		
		
		
		
\chapter{Background}
\section{Why is it important?}
\section{Where are the challenges?}
\section{What am I going to solve, and why?}
\section{Related work}

\section{Machine Learning}
Testing a cite:\\
\textit{ A computer program is said to learn from experience E with respect to 
some class of tasks T and performance measure P, if its performance at
tasks in T, as measured by P, improves with the experience E. } 
\cite{MitchellTomM1997Ml}
\subsection{Tasks}
\label{chap:Tasks}
\begin{itemize}
 \item Classification
 \item regression 
 \item transcription/translation
 \item de-noising /finding missing inputs
\end{itemize}
\subsection{The rate of success}
What is a good result, how to measure?\\
\textbf{FP,TN,FN,TP}\\


%\subsection{deep vs shallow}		
\subsection{supervised vs unsupervised}
What it means to be S/US.\\
Something about the kind of experience allowed during the learning process.


\section{Unsupervised}
noe med å dele i grupper?
Experience the dataset containing many features, and finds useful properties of the structures. 
\textit{\textbf{Unsupervised learning algorithms} experience a dataset containing manyfeatures, then learn useful properties of the structure of this dataset. In the contextof deep learning, we usually want to learn the entire probability distribution thatgenerated a dataset, whether explicitly, as in density estimation, or implicitly, fortasks like synthesis or denoising. Some other unsupervised learning algorithmsperform other roles, like clustering, which consists of dividing the dataset intoclusters of similar examples.}
\cite{Goodfellow-et-al-2016}


\subsection{Approaches to unsupervised learning}
look at the \autoref{chap:Tasks} to see what applies to the unsupervised.

\subsection{Deep Unsupervised learning}
\subsection{more}


	      
		
		
		
		
		
		
		
		
		
		
		
		
		
		
		
		
		
		
		
		
		
		
		
		
		
\part{The project}

\chapter{Planning the project}

\part{Conclusion}

\chapter{Results}

\backmatter{}

\printbibliography

\end{document}
