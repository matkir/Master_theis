\documentclass[a4paper,english]{ifimaster}

\usepackage[utf8]{inputenc}
\usepackage{babel,duomasterforside}
\usepackage{csquotes}
\usepackage{hyperref}

\usepackage[backend=biber,style=authoryear]{biblatex}
\addbibresource{bibliography.bib}





\title{Learning}
\subtitle{Subtitle}
\author{Mathias Kirkerød}

\begin{document}
\duoforside[dept={Department of Informatics},
program={Informatics: Language and Communication},
long]

\frontmatter{}
\chapter*{Abstract}

\tableofcontents{}
\listoffigures{}
\listoftables{}

\chapter*{Preface}

\mainmatter{}






















\part{Introduction}
\chapter{Introduction}
	\section{Background and Motivation}
		Something about cancer and current treatments 
		* Increasing cancer rate
		* 2 main options (colonoscopy MRI)
		* the 3rd option
		* CAD ACD (computer aided diagnosis, Automated computer diagnosis)
		* Simulas contribution
		* Simulas EIR
		* Now that we got a lot of tests, why not unsupervised
	
	\section{Goal / Problem}
		* We know that we can get some results using a neural network
		* Can this be done unsupervised?
		* Can it be done in a fashion that is better than S-ML
		 
		
		
	\section{Scope and Limitations}
		* Something about earlier research already got far, so the scope is mainly unsupervised deep learning.
		* (and how to generalise it?)
		
	
		

	\section{Outline}
	The rest of the thesis is structed as follows:
	
	\textbf{Chapter 2 - Background}\\
	*talk about cancer
	*talk about machine learning.
	\textbf{Chapter 3 - Me doing stuff}\\
	\textbf{Chapter 4 - Me got and presented result}\\
	\textbf{Chapter 5 - Me saying result was good A+}\\
	
		
\chapter{Background}
	\section{Cancer}
	  \subsection{regular colonoscopy/ Gastroscopy}
	  \subsection{Pillcam}
	  
	  
	\section{Machine Learning}



	Testing a cite:\\

	\textit{ A computer program is said to learn from experience E with respect to 
	some class of tasks T and performance measure P, if its performance at
	tasks in T, as measured by P, improves with the experience E. } 
	\cite{MitchellTomM1997Ml}
	  \subsection{Tasks (other better word goes here)}
\label{chap:Tasks}
\begin{itemize}
 \item Classification
 \item regression 
 \item transcription/translation
 \item de-noising /finding missing inputs
\end{itemize}
\subsection{The rate of success}
What is a good result, how to measure?\\
\textbf{FP,TN,FN,TP}\\


%\subsection{deep vs shallow}		
\section{supervised vs unsupervised}
What it means to be S/US.\\
Something about the kind of experience allowed during the learning process.


\section{Unsupervised}
noe med å dele i grupper?
Experience the dataset containing many features, and finds useful properties of the structures. 
\textit{\textbf{Unsupervised learning algorithms} experience a dataset containing manyfeatures, then learn useful properties of the structure of this dataset. In the contextof deep learning, we usually want to learn the entire probability distribution thatgenerated a dataset, whether explicitly, as in density estimation, or implicitly, fortasks like synthesis or denoising. Some other unsupervised learning algorithmsperform other roles, like clustering, which consists of dividing the dataset intoclusters of similar examples.}
\cite{Goodfellow-et-al-2016}


\subsection{Approaches to unsupervised learning}
look at the \autoref{chap:Tasks} to see what applies to the unsupervised.

\subsection{Deep Unsupervised learning}
\subsection{more}
\section{Related work}


	      
		
		
		
		
		
		
		
		
		
		
		
		
		
		
		
		
		
		
		
		
		
		
		
		
		
\part{The project}

\chapter{Planning the project}

\part{Conclusion}

\chapter{Results}

\backmatter{}

\printbibliography

\end{document}
