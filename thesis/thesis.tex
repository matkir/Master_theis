\documentclass[a4paper,english]{ifimaster}

\usepackage[utf8]{inputenc}
\usepackage{babel,duomasterforside}
\usepackage{csquotes}
\usepackage{hyperref}

\usepackage[backend=biber,style=authoryear]{biblatex}
\addbibresource{bibliography.bib}





\title{Learning}
\subtitle{Subtitle}
\author{Mathias Kirkerød}

\begin{document}
\duoforside[dept={Department of Informatics},
program={Informatics: Language and Communication},
long]

\frontmatter{}
\chapter*{Abstract}

\tableofcontents{}
\listoffigures{}
\listoftables{}

\chapter*{Preface}

\mainmatter{}






















\part{Introduction}
\chapter{Introduction}
	\section{Background and Motivation}
	\textbf{Cancer}
	Cancer is, today, the second leading cause of death in the world, only behind 
	cardiovascular diseases.\\ %TODO CITE
	\textit{The most common types of cancer in males are lung cancer, prostate cancer, colorectal cancer and stomach cancer.\cite{stewart2014world}}
	\vspace{10px}
	
	\textbf{Statistics}
	The western (or modern) world has been in a battle against cancer, and despite a 
	lot of new cures/innovations it is still one of the deadliest killers in the world. 
	
	
	\vspace{10px}
	\textbf{colorectal cancer}
	You can get cancer in every major organ, but some types of cancer are more common than others.
	For instance cancer in the gastrointestinal tract (GI) is one of the more common places 
	to get cancer. This is just behind x, and it has a mortality rate of x in the first y years. %TODO CITE 
	We often call this 5 year survival rate for z. This is the standard way to measure the life expectancy of a patient diagnosed with cancer. 
	
	
	\vspace{10px}
	\textbf{polyps}
	The colorectal cancer often starts in polyps. Polyps are, polyps do.
	\\
	
	\vspace{10px}
	\textbf{preventative matters and early detection}
	A good way to fight cancer is to detect and remove it early, or some times remove areas with a high chance of  getting cancer.
	We classify cancer in to x stages, and the stage the patient are in often determines the chance you have for survival. 
	In general, the earlier you find the cancer, the more likely it is that the patient will survive. 
	And as mentioned above, the colorectal cancer often starts in these polyps. A crucial stage to prevent cancer lies in the 
	early removal of there polyps.
	Reports shows x about this %TODO find Reports
	
	*4 stages maybe?
	*early detection
	*survival rate
	
		
	Because of this the ability to find, and remove colorectal polyps is great for preventing cancer in the GI tract. 
	
	
	\vspace{10px}
	\textbf{colonoscopy/Ontonoscopy}
	In the most common way to look for polyps in the GI tract is to use a medical team, and perform a colonoscopy or Ontonoscopy
	colonoscopy is preformed with a camera.....\\
	Onoskopy is the same procedure, only the camera is inserted orally. \\
	\textbf{Advantages}
	  * Accurate 
	  *
	\vspace{10px}
	\textbf{Disadvantages}
	  *expensive 
	
	\vspace{10px}
	\textbf{MRI}
	
	\vspace{10px}
	\textbf{pillcam}
	
	In the last 3-4 years there have been testing and development on the pillcam project EIR. Machine learning has, through 
	many of the earlier projects, got the detection rate for the polyps up to x\% %TODO cite
	
	
	\vspace{10px}
	\textbf{Simulas contribution}
	Simulas EIR
		

		
	* CAD ACD (computer aided diagnosis, Automated computer diagnosis)
	
	\section{Goal / Problem}
		* Now that we got a lot of tests, why not unsupervised
		As mentioned, simula research centre has done a lot of testing on the pillcam project.
		  
		* We know that we can get some results using a neural network
		* Can this be done unsupervised?
		* Can it be done in a fashion that is better than S-ML
		 
		
		
	\section{Scope and Limitations}
		* Something about earlier research already got far, so the scope is mainly unsupervised deep learning.
		* (and how to generalise it?)
		
	
		

	\section{Outline}
	The rest of the thesis is structed as follows:
	
	\textbf{Chapter 2 - Background}\\
	*talk about cancer
	*talk about machine learning.
	*how to use ML on the pillcam video?
	\textbf{Chapter 3 - Me doing stuff}\\
	\textbf{Chapter 4 - Me got and presented result}\\
	\textbf{Chapter 5 - Me saying result was good A+}\\
	
		
\chapter{Background}
	\section{Cancer}
	  \subsection{regular colonoscopy/ Gastroscopy}
	  \subsection{Pillcam}
	  
	  
	\section{Machine Learning}
	Machine learning is a very broad term, but can i short be summarised by:\\
	\vspace{10px}
	
	\textit{ A computer program is said to learn from experience E with respect to 
	some class of tasks T and performance measure P, if its performance at
	tasks in T, as measured by P, improves with the experience E. } 
	\cite{MitchellTomM1997Ml}\\
	
	\vspace{10px}
	Here we have a couple of parameters:\\
	\textbf{E} text about p\\
	\textbf{T} text about p\\
	\textbf{P} text about p\\
	
	From this we see that the goal of machine learning is to improve some performance P with experience.
	\textbf{might here talk about different tasks ML can do?}
	
	  \subsection{Supervised \& Unsupervised machine learning}
	  We often divide machine learning in to two (diffuse) categories: supervised and unsupervised.\\
	  \textbf{Supervised learning:} is the act of training with data that has an answer. The learning algorithm cant get supervision while 
	  training on the task. %TODO more
	  
	  \textbf{Unsupervised learning:} is the act of training without any supervision, on the sense that we do not give the algorithm the answer
	  to the training data set. %TODO more 
	  
	  In the description of supervised vs unsupervised we looked at a specific branch of machine learning: Classification. Classification is, as the name implies, the task of 
	  getting data sorted in to groups of similarity. 
	  
	  
	  \begin{itemize}
	    \item Classification
	    \item regression 
	    \item transcription/translation
	    \item de-noising /finding missing inputs
	  \end{itemize}
	  
	  Now that we have the definintion of machine learning we focus on the task at hand; finding polyps. In an ideal world we have a
	  Classification problem with only two classes: Non-polyp and polyp. 
	  
	  \begin{itemize}
	    \item SVM 
	    \item CNN 
	    \item random forests
	    \item knn
	  \end{itemize}
	  
	  \subsection{CNN}
	  \subsubsection{UCNN?}
	  
	  
	  
	  
	  
	
	
	
	
	  \subsection{Tasks (other better word goes here)}
\label{chap:Tasks}

\subsection{The rate of success}
What is a good result, how to measure?\\
\textbf{FP,TN,FN,TP}\\


%\subsection{deep vs shallow}		
\section{supervised vs unsupervised}
What it means to be S/US.\\
Something about the kind of experience allowed during the learning process.


\section{Unsupervised}
noe med å dele i grupper?
Experience the dataset containing many features, and finds useful properties of the structures. 
\textit{\textbf{Unsupervised learning algorithms} experience a dataset containing manyfeatures, then learn useful properties of the structure of this dataset. In the contextof deep learning, we usually want to learn the entire probability distribution thatgenerated a dataset, whether explicitly, as in density estimation, or implicitly, fortasks like synthesis or denoising. Some other unsupervised learning algorithmsperform other roles, like clustering, which consists of dividing the dataset intoclusters of similar examples.}
\cite{Goodfellow-et-al-2016}


\subsection{Approaches to unsupervised learning}
look at the \autoref{chap:Tasks} to see what applies to the unsupervised.

\subsection{Deep Unsupervised learning}
\subsection{more}
\section{Related work}


	      
		
		
		
		
		
		
		
		
		
		
		
		
		
		
		
		
		
		
		
		
		
		
		
		
		
\part{The project}

\chapter{Planning the project}
\section{Using Generative Adversarial Networks for enhancement and prediction}
Generative Adversarial Networks (GAN) was proposed by Goodfellow in 2014. \cite{GoodfellowGAN} %TODO fix cite
GANs is a specialised network consisting of generally two neural networks working against each other. 
The two networks are often called discriminator and generator. The generators job is to make data similar to the training data, and discriminator has the job if finding fake (generated) data.
%TODO insert picture here?

The GAN presented by Goodfellow 2014 is a simple

\subsection{DC-GAN}
  convolutional, good for images.  
\subsection{CC-GAN}
  removing green squares



\part{Conclusion}

\chapter{Results}

\backmatter{}

\printbibliography

\end{document}
